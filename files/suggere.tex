
\documentclass[../main.tex]{subfile}

\begin{document}

\section{Le rêve insinué}

La base même d'un rêve est les sensations qu'il produit pour le sujet. Ces
sensations sont au-délà des simples informations sur le monde extérieur
auquelles nous sommes habitués. Elles présentes certaines caractéristiques qui
sont intéressantes à reproduire pour imiter cet effet.

Cependant, cette idée est bien plus difficile à réaliser qu'il n'y paraît. Un
spectateur n'a a sa disposition que \emph{la vue} et \emph{l'ouïe} comme canaux
de communication avec l'univers du film regardé. Pourtant, au sein d'un rêve,
le sujet perçoit une grande variété de sensations différentes. C'est donc un
problème majeur pour la reproduction d'un rêve dans une \oe{}uvre
cinématographique.

\subsection{Aspect visuel}

\subsubsection{Mélange de couleurs}
\subsubsection{Troubles visuels}

\subsection{Aspect auditif}

\subsection{Symbolisme}

\end{document}
