\documentclass[../main.tex]{subfile}

\begin{document}

\section{Introduction}

Le rêve est une expérience sensorielle relatant souvent des événements rééls à
travers une représentation fictive. Dans le cadre du cinéma, c'est un moyen
permettant "de révéler sans affecter" le cours naturel du scénario. La
difficulté de mise en scène un rêve sur grand écran vient surtout du contexte
sensoriel partagé avec la réalité représentée, car ces deux ne sont qu'oeuvres
visuelles et auditives.

{\bfseries \large Alors quels techniques un réalisateur peut-il utiliser pour
    différencier les mondes du rêve de la réalité au cinéma ? \\}

\end{document}
